\documentclass{article}
\usepackage[
    pdftex,
    pdfauthor={Jorge C. Henríquez Tolhuysen},
    pdftitle={Curriculum Vitae de Jorge C. Henríquez Tolhuysen},
    pdfsubject={Curriculum Vitae},
    pdfkeywords={
        Curriculum Vitae, 
        Jorge Henríquez, 
        Estudios, 
        Experiencia,
        Sistemas Desarrollados,
        Lenguajes de programación, 
        Java, 
        Visual Basic 6.0, 
        Visual Basic .Net, 
        C\#, 
        Php,
        Threejs, 
        Javascript, 
        Typescript, 
        Tailwindcss,
        Html,
        Css,
        Json,
        Svg,
        Dwg,
        Bim,
    },
    pdfproducer={Latex con hyperref},
    pdfcreator={pdflatex}
]{hyperref}                 % Paquete para hipervínculos
\usepackage[T1]{fontenc}    % Set the font (output) encodings
\usepackage{tgbonum}
\usepackage[spanish]{babel} % Spanish-specific commands
\usepackage[
    paperwidth=21.59cm,     % Set the width of the paper
    paperheight=27.94cm,    % Set the height and width of the paper
    includehead,
    nomarginpar,            % We don't want any margin paragraphs
    textwidth=18cm,         % Set \textwidth to 10cm
    textheight=21.5cm,
    headheight=87.93pt,      % Set \headheight to 83.9 pt
]{geometry}
\usepackage{xcolor}
\usepackage{fancyhdr}
\author{Jorge C. Henríquez Tolhuysen}
\date{\today}
\begin{document}
\pagestyle{fancy}           % Set the page style to "fancy"...
\fancyhead{}                % clear all header fields
\fancyfoot{}                % clear all footer fields
\fancyhead[C]{
    \colorbox{lightgray}{%
        \parbox{0.98\headwidth}{
            \centering
            \vspace{2mm}
            \textbf{\LARGE{JORGE CRISTIAN HENRÍQUEZ TOLHUYSEN}}\\
            \vspace{2mm}
            \hrule height 2pt
            \vspace{2mm}
            \textbf{Ingeniero Electrónico de la Universidad Técnica Federico Santa María.}\\
            \textbf{Contador Auditor de la Universidad Tecnológica de Chile, INACAP.}\\
            \vspace{2mm}
            \hrule
            \vspace{1mm}
            \textit{Número Celular: \href{https://wa.me/56941191371}{+56 9 4119 1371}, Dirección E-Mail: \href{mailto:jorge.henriquez@me.com}{jorge.henriquez@me.com}}
            \vspace{1mm}
        }
    }%
}
\fancyfoot[R]{\thepage} % footer for the right side
\fancyfoot[L]{Jorge Henríquez Tolhuysen, \today, creado con \LaTeX} % footer for the left side

\section{Competencias Técnicas}
\label{sec:competencias-section}
Instalación y configuración de S.O. de Escritorio y Servidores, Configuración de Redes y servicios relacionados, Manejo Avanzado de TCP/IP, Desarrollo de Sistemas en diversos Lenguajes de Programación.

\section{Caracteristicas Personales}
\label{sec:caracteristicas-section}
Leal, Analítico, Proactivo, Riguroso, Gentil, Alta Motivación por el Trabajo, Cooperador, Seguro, Altamente adaptable al cambio, Aprendizaje rápido.

\section{Estudios/Educación}
\label{sec:estudios-section}
\begin{itemize}
    \item Desde 1989 al 1994, \textbf{Ingeniería Electrónica}, título obtenido en la \href{https://www.usm.cl}{\textbf{Universidad Técnica Federico Santa María}} en Valparaiso, Chile, Ayudante para las asignaturas de Programación de Computadores en Pascal y Telefonía.
    \item Desde 2013 al 2016, \textbf{Contabilidad y Auditoría}, Todas las asignaturas aprobadas, sin titulación, práctica profesional realizada fuera de plazo, estudios realizados en la \href{https://www.inacap.cl}{\textbf{Universidad Tecnológica de Chile, INACAP}}, dos años consecutivos reconocido como el mejor estudiante del área de administración y negocios.
\end{itemize}

\section{Certificaciones}
\label{sec:certificaciones-section}
\begin{itemize}
    \item 2006, Sun Certified Java Programmer de Sun Microsystems, con 88\% rendimiento.
\end{itemize}

\section{Experiencia Laboral}
\label{sec:laboral-section}
\begin{itemize}
    \item 2026, en \textbf{ALVAREZ \& ASOCIADOS S.p.A.} como Consultor Independiente, Migra aplicación para el Escritorio de Windows, escrita en C\#, con .Net Framework 4.78 y usando Windows Forms a versión con .Net 8.0 y WinUi 3.
    \begin{description}
        \item[Tecnologías relacionadas:] \textit{Visual Studio Professional 2026, C\#, .Net 8.0, WinUI 3, Html, Javascript, Css.}
    \end{description}
    \item Desde 2021 al 2025, en \href{https://www.get-fm.de}{\textbf{FACILITY CONSULTANTS GmbH}} como Consultor Independiente, Desarrolló una Clase PHP que toma un Archivo AutoCAD 2D DWG y genera código SVG. Programó una Libreria JavaScript que acepta un archivo BIM y produce un renderizado 3D del mismo, ambas son usadas en el mismo sitio web.
    \begin{description}
        \item[Tecnologías relacionadas:] \textit{Visual Studio Code, Php, Javascript, Typescript, Tailwindcss, Threejs.}
    \end{description}
    \item Desde 2021 al 2025, en \textbf{ALVAREZ \& ASOCIADOS S.p.A.} como Consultor Independiente, Construyó una aplicación para el Escritorio de Windows, escrita en C\#, que hace web scraping de un ERP y genera salidas de Balances y Estados de Resultados en pantalla y archivos Excel.
    \begin{description}
        \item[Tecnologías relacionadas:] \textit{Visual Studio Professional 2022, Webscraping, C\#.}
    \end{description}
    \newpage
    \item Desde 2021 al 2025, en \href{https://stratex.cl}{\textbf{STRATEX CONSULTORES}} como Consultor Independiente, Programó un Complemento Excel, escrito en C\#, para Importar y Fusionar Archivos de Texto de manera desatendida.
    \begin{description} 
        \item[Tecnologías relacionadas:] \textit{Visual Studio Professional 2022, Integración con Excel, C\#}.
    \end{description}
    \item Desde 2021 al 2024, en \href{https://www.khipu.com}{\textbf{KHIPU S.p.A.}} como Ingeniero de Software Senior en area backend.
    \begin{description} 
        \item[Tecnologías relacionadas:] \textit{IntelliJ, Java, Postgresql, Bitbucket, Git, Terraform, IBM Cloud, Docker.}
    \end{description}
    \item Desde 2019 al 2021, en \href{https://www.comciencia.cl/}{\textbf{COMCIENCIA INFORMÁTICA S.p.A.}} como Desarrollador Full Stack Senior, para el software \href{https://www.s-verein.de/}{S-Verein}, COMCIENCIA es una Compañia con Base en Chile de propiedad Alemana. 
    \begin{description} 
        \item[Tecnologías relacionadas:] \textit{Php, Oracle DB, Mysql DB, Maria DB, PhpStorm, Youtrack, Svn, Git, Linux, Aws.}
    \end{description}
    \item Desde 2010 al 2019, en \textbf{SERVIMAQ S.A.} como Consultor Independiente, Configurando y Administrando Windows Server y SQL Server. Administrando Firewall Watchguard y Desarrollando Software.
    \begin{description} 
        \item[Tecnologías relacionadas:] \textit{Windows Server, Ms Sql Server, Firewall Watchguard, Genexus, Visual Basic .NET, Java}
    \end{description}
    \item Desde 2003 al 2025, en \href{https://trucaoberries.com/es/inicio/}{\textbf{TRUCAO BERRIES}} como Consultor Independiente, Configurando y Administrando Windows Server, SQL Server y Softland Erp. Administrando Firewall Watchguard. Desarrollando Software.
    \begin{description} 
        \item[Tecnologías relacionadas:] \textit{Windows Server, Ms Sql Server, Softland Erp, Firewall Watchguard, Visual Studio Professional, Visual Basic .NET, Android Studio, Java}.
    \end{description}
    \item Desde 2013 al 2021, en \href{https://www.integra-chile.cl}{\textbf{INTEGRA CHILE}} como Consultor Independiente, Configurando y Administrando Windows Server y SQL Server. Administrando Firewall Watchguard/Fortinet.
    \begin{description} 
        \item[Tecnologías relacionadas:] \textit{Windows Server, Ms Sql Server, Firewall Watchguard, Firewall Fortinet.}
    \end{description}
    \item Desde 2016 al 2017, en \href{https://www.liceoraac.cl}{\textbf{LICEO R.A.A.C.}} como Profesor para las asignaturas de Redes y Comunicaciones.
    \item Desde 2003 al 2011, en \textbf{TERRAGRICOLA} como consultor Independiente, Configurando y administrando Servidores Windows/Netware/Linux. Administrando Firewall Watchguard.
        \begin{description} 
        \item[Tecnologías relacionadas:] \textit{Windows Server, Novell Netware, Linux, Firewall Watchguard.}
    \end{description}
    \item Desde 2008 al 2009, en \href{https://www.uach.cl}{\textbf{UNIVERSIDAD AUSTRAL DE CHILE}} como Profesor Asistente para las asignaturas relacionadas con Ms Office en la Escuela de Medicina. Desarrollo de software.
    \begin{description} 
        \item[Tecnologías relacionadas:] \textit{Visual Studio Professional, Visual Basic .NET, Adobe FLEX.}
    \end{description}
    \item Desde 2005 al 2006, en \href{https://portal.inacap.cl}{\textbf{INACAP}} como Profesor en varias asignaturas para la carrera de Ingeniería Informática en las sedes de Osorno y Puerto Montt.
    \item Desde 2004 al 2005, en \href{https://www.ulagos.cl}{\textbf{UNIVERSIDAD DE LOS LAGOS}} como Profesor en varias asignaturas para la carrera de Ingeniería Informática en las sedes de Osorno y Puerto Montt.
    \item Desde 2003 al 2005, en \textbf{SINGA S.A.} como Lider de Proyecto y Desarrollador Senior para el proyecto SIPP de Marine Harvest Puerto Montt.
    \begin{description} 
        \item[Tecnologías relacionadas:] \textit{GENEXUS generando código Visual Basic 6.0.}
    \end{description}
    \newpage
    \item El 2003, en \textbf{INNOVA CHILE} como Técnico de campo para las estaciones de combustible ESSO.
    \item El 2002, en \href{https://portal.inacap.cl}{\textbf{INACAP}} como Profesor en varias asignaturas para la carrera de Ingeniería Informática en la sede Osorno.
    \item Desde 2001 al 2002, en \textbf{TECNOLÓGICO DE LA UNIVERSIDAD DE LOS LAGOS} como Profesor en varias asignaturas para la carrera de Técnico en Informática.
    \item Desde 1999 al 2002, en \href{https://telsur.cl}{\textbf{TELEFÓNICA DEL SUR}} como Ingeniero de Proyectos para el Area Comercial, Zonal Osorno.
    \item Desde 1997 al 1998, en \textbf{CDI DE LA UNIVERSIDAD DE LOS LAGOS} como Profesor en varias asignaturas la carrera de Ingeniería Informática.
\end{itemize}

\section{Experiencia relevante en desarrollo de software}
\label{sec:desarrollo-section}
\begin{itemize}
    \item En el 2026, Migra aplicación escrita en C\#, usando .Net Framework 4.78 y Windows Forms a .Net 8.0 y WinUI 3.
    \begin{description}
        \item[Nombre del Software:] \textbf{SICE.NEXT.}
        \item[Desarrollado para:] \textit{ALVAREZ \& ASOCIADOS S.p.A., Osorno, Chile.}
        \item[Posición:] \textit{Analista y Desarrollador.}
        \item[Herramientas:] \textit{Visual Studio 2026, C\#, .Net 8.0, WinUI 3, WebView, Excel, Ceres ERP.}
        \item[Labores:] \textit{Análisis, Diseño y Desarrollo.}
    \end{description}
    \item Desde 2021 al 2025, Desarrollo de clase Php para la conversión de archivos Autocad DWG a datos SVG, Desarrollo de libreria Javascript para la conversión de archivos BIM a formato IFC y su posterior renderizado 3D.
    \begin{description}
        \item[Nombre del Software:] \textbf{GETFM.}
        \item[Desarrollado para:] \textit{FACILITY CONSULTANTS GmbH, Nufringen, Alemania.}
        \item[Posición:] \textit{Analista y Desarrollador.}
        \item[Herramientas:] \textit{Visual Studio Code, Php, Javascript, Tailwindcss.}
        \item[Labores:] \textit{Análisis, Diseño y Desarrollo.}
    \end{description}
    \item Desde 2021 al 2025, Desarrollo de software para windows, que realiza Webscraping a sistema ERP en la web y genera Balances contables y Estados de resultados en formatos personalizados.  
    \begin{description}
        \item[Nombre del Software:] \textbf{SICE.}
        \item[Desarrollado para:] \textit{ALVAREZ \& ASOCIADOS S.p.A., Osorno, Chile.}
        \item[Posición:] \textit{Analista y Desarrollador.}
        \item[Herramientas:] \textit{Visual Studio 2022, C\#, Excel, Ceres ERP.}
        \item[Labores:] \textit{Análisis, Diseño y Desarrollo.}
    \end{description}
    \newpage
    \item Desde 2021 al 2025, Desarrollo de complemento para Excel que permite importar Libros de Compras, Ventas y Honorarios generados por SAP.  
    \begin{description}
        \item[Nombre del Software:] \textbf{EXAISTRATEX.}
        \item[Desarrollado para:] \textit{STRATEX CONSULTORES, Osorno, Chile.}
        \item[Posición:] \textit{Analista y Desarrollador.}
        \item[Herramientas:] \textit{Visual Studio 2022, C\#, Excel.}
        \item[Labores:] \textit{Análisis, Diseño y Desarrollo.}
    \end{description}
    \item Desde 2013 al 2018, Desarrollo de Software e Integración con terminal TRANSBANK lector de tarjetas de debito/crédito para su producto POS INTEGRADO con un sistema WEB.
    \begin{description}
        \item[Nombre del Software:] \textbf{SIPOS.}
        \item[Desarrollado para:] \textit{VITAMINA WORK LIFE, Santiago, Chile.}
        \item[Posición:] \textit{Analista y Desarrollador.}
        \item[Herramientas:] \textit{Visual Studio 2017, Visual Basic .NET, ASP .NET.}
        \item[Labores:] \textit{Análisis, Diseño y Desarrollo.}
    \end{description}    
    \item Desde 2013 al 2018, Desarrollo de Software e Integración de ERP MANAGER con software de terceros para la emision de documentos tributarios electrónicos (DTE).
    \begin{description}
        \item[Nombre del Software:] \textbf{SIDTE.}
        \item[Desarrollado para:] \textit{SERVIMAQ S.A., Osorno, Chile.}
        \item[Posición:] \textit{Analista y Desarrollador.}
        \item[Herramientas:] \textit{Visual Studio 2017, Visual Basic .NET, Xml, Dte}
        \item[Labores:] \textit{Análisis, Diseño y Desarrollo.}
    \end{description}
    \item Desde 2012 al 2018, Desarrollo de sistema para planta de empacado de arandanos, integrando balanzas/romanas estaticas y software de terceros, culminando con un sistema autonomo. 
    \begin{description}
        \item[Nombre del Software:] \textbf{SIP.}
        \item[Desarrollado para:] \textit{TRUCAO BERRIES, Rio Negro, Chile.}
        \item[Posición:] \textit{Analista y Desarrollador.}
        \item[Herramientas:] \textit{Visual Studio 2019, Visual Basic .NET, Rs-232, Tcp-Ip}
        \item[Labores:] \textit{Análisis, Diseño y Desarrollo.}
    \end{description}
    \item Desde 2011 al 2018, Desarrollo, mantenimiento y evolución de software para control de stock y agendamiento de citas para el cumplimiento de la certificación KODAWARI para concesionarios TOYOTA. 
    \begin{description}
        \item[Nombre del Software:] \textbf{KODAWARI.}
        \item[Desarrollado para:] \textit{SERVIMAQ S.A.,Osorno, Chile.}
        \item[Posición:] \textit{Analista y Desarrollador.}
        \item[Herramientas:] \textit{Genexus 9, Visual Basic 6}
        \item[Labores:] \textit{Análisis, Diseño y Desarrollo.}
    \end{description}
    \newpage
    \item Desde 2003 al 2018, Desarrollo, mantenimiento y evolución de un sistema integrado Windows/Android para la recolección de información en terreno para la cosecha de arandanos, utilizando terminales de datos Honeywell, teléfonos celulares Samsung y un sistema de pago de nóminas. 
    \begin{description}
        \item[Nombre del Software:] \textbf{SIC.}
        \item[Desarrollado para:] \textit{TRUCAO BERRIES, Rio Negro, Chile.}
        \item[Posición:] \textit{Analista y Desarrollador.}
        \item[Herramientas:] \textit{Visual Studio 2017, Visual Basic .NET, Android Studio, Java, Terminales Honeywell, Windows Embedded, Teléfonos Celulares Samsung, Códigos de Barra, Códigos QR, Impresoras Zebra}
        \item[Labores:] \textit{Análisis, Diseño y Desarrollo.}
    \end{description}
    \item Desde 2015 al 2016, Desarrollo de sistema integrado Windows/Android para la recolección de información en terreno de despachos de petroleo y la emisión del correspondiente documento tributario electrónico generado a travez de la automatización de software de terceros. 
    \begin{description}
        \item[Nombre del Software:] \textbf{SIRC.}
        \item[Desarrollado para:] \textit{CONSTRUCTORA HARR S.A., Osorno, Chile.}
        \item[Posición:] \textit{Analista y Desarrollador.}
        \item[Herramientas:] \textit{Visual Studio 2017, Visual Basic .NET, Android Studio, Java, Transtecnia Contabilidad, Tcp-Ip}
        \item[Labores:] \textit{Análisis, Diseño y Desarrollo.}
    \end{description}
    \item En el 2016, Software para la extracción de datos para Balances Contables desde los datos ingresados en ERP FLEXLINE.
    \begin{description}
        \item[Nombre del Software:] \textbf{SIFLEX.}
        \item[Desarrollado para:] \textit{SALMONES CALETA BAY, Osorno, Chile.}
        \item[Posición:] \textit{Analista y Desarrollador.}
        \item[Herramientas:] \textit{Visual Studio 2017, Visual Basic .NET, ERP Flexline}
        \item[Labores:] \textit{Análisis, Diseño y Desarrollo.}
    \end{description}
    \item Desde 2003 al 2005, Desarrollo para software de control de producción de plantas de proceso de salmón, integrando comunicación con balanzas estáticas y dinámicas, para el calibrado de materia prima, producto términado y registro de embalajes, asi como para los terminales de recolección de datos Honeywell. 
    \begin{description}
        \item[Nombre del Software:] \textbf{SIPP.}
        \item[Desarrollado para:] \textit{MARINE HARVEST, Puerto Montt, Chile.}
        \item[Posición:] \textit{Analista y Desarrollador.}
        \item[Herramientas:] \textit{Genexus 9, Visual Basic 6, Rs-232, Terminales Honeywell, Windows Embedded, Códigos de Barra, Impresoras Zebra}
        \item[Labores:] \textit{Liderazgo, Análisis, Diseño y Desarrollo.}
    \end{description}
\end{itemize}
\end{document}